\documentclass[12pt,a4paper]{article}
\usepackage[margin=1.25in]{geometry}
\usepackage{fancyhdr} % fancy header
\pagestyle{fancy} % so fancy
\usepackage[russian,english]{babel} % for russian letters
\usepackage{tipa} % for IPA symbols
\usepackage[round]{natbib} % bibliography
\usepackage{graphicx} % for importing graphics / figures
\usepackage{booktabs} % publication-worthy tables
\usepackage{adjustbox} % makes tables fit nicely on the page


\lhead{Josh MEYER}
\rhead{Practical Trial for LSCP Position}
\cfoot{} %% make empty to get rid of the page number %% \cfoot{Page \thepage}
\renewcommand{\footrulewidth}{0.4pt} %% this puts a fancy line at the footer


\begin{document}

\subsection*{Code}

To run the submitted code, use the \texttt{run.py} file as such:

\texttt{\$ python run.py small 1}

Here, \texttt{small} is the name of the data directory, and \texttt{1} is the level of verbosity. Currently verbosity may be set either on or off (0 or 1).

\subsection*{Introduction}

The following relates to the practical trial on January 15, 2018. The applicant (Josh) was sent a directory of code and data relating to an implementation of the AX discrimination score. This writeup is to accompany the code modified by the applicant.


\subsection*{What I did}

\begin{itemize}
  
\item \textbf{Variable Renaming} The code was modified such that items \texttt{A}, \texttt{B}, and \texttt{X} are evident. The exception to this is the internal variables of the added function \texttt{get\_DTW()}, which uses \texttt{X} and \texttt{Y} so as to be agnostic to input. 

\item \textbf{Code Correction} The \texttt{if} statement in the choosing of item \texttt{X} was modified to avoid picking item \texttt{A} again.

              %% \texttt{ for data\_item\_X in \[e for e in files if \( np.int\(e\[-1\]\) == label\_A and e \!\= data\_item\_A \)\]\:}
\item \textbf{Improve Structure of Code} The repeated \texttt{DTW} calculation used to estimate the ABX measure was double hard-coded. This modified code makes the calculation a stand-alone function.

\item \textbf{Document Code} Commenting was used to break the code into logical parts, such as data loading, and DTW calculation.

\item \textbf{Profile Code} Time stamps are used to show length of time used by I/O or calculations.

\end{itemize}

\subsection*{What I did not do}

\textbf{Optimize Code for Time} Calculating DTW with Numpy would surely be faster than mutliple nested forloops, but I did not manage to implement it.

\subsection*{Comments}

The major problems I see with this code relate to scaling to larger files (i.e. more dimensions per item), and more files. Having more files will tax the program on I/O time, and more dimensions will tax the distance calculations. To solve the former problem, the data can be saved on disc in a format best for reading and writing, modified for Python (i.e. there may be better formats than HDF5). As for computation, implementing everything with linear algebra libraries would be ideal.

\end{document}
