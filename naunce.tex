\documentclass[12pt,a4paper]{article}
\usepackage[margin=1.25in]{geometry}
\usepackage{fancyhdr} % fancy header
\pagestyle{fancy} % so fancy
\usepackage[russian,english]{babel} % for russian letters
\usepackage{tipa} % for IPA symbols
\usepackage[round]{natbib} % bibliography
\usepackage{graphicx} % for importing graphics / figures
\usepackage{booktabs} % publication-worthy tables
\usepackage{adjustbox} % makes tables fit nicely on the page
\usepackage{hyperref}

\lhead{Joshua MEYER}
\rhead{Interest: Intern - Deep learning for text classification}
\cfoot{} %% make empty to get rid of the page number %% \cfoot{Page \thepage}
\renewcommand{\footrulewidth}{0.4pt} %% this puts a fancy line at the footer


\begin{document}


\subsection*{To Whom it May Concern:}

My name is Joshua Meyer, and I am currently a PhD candidate at the University of Arizona.

As a current student working on acoustic modeling for automatic speech recognition, Nuance's current internship position in Deep learning for text classification significantly interests me because internship will allow me to further my practical skills in NLP and Deep learning.

On a daily basis, I spend my time working with audio and deep nets (ie. recurrent and convolutional). I am competent in Kaldi and comfortable with CMU-Sphinx, and I have produced tutorials and documentation for both (my Kaldi documentation has even been reviewed by Dan Povey).

Even though text classification is not my main area of research, I believe I would be a good fit for this position at Nuance because I have already authored (and presented) a tutorial on text classification with TensorFlow:

\begin{center}
\href{<http://jrmeyer.github.io/tutorial/2016/02/01/TensorFlow-Tutorial.html>}{http://jrmeyer.github.io/tutorial/2016/02/01/TensorFlow-Tutorial.html} \\
\end{center}

That is to say, I am very familiar with machine learning in general, Deep learning in particular, as well as NLP. However, the practical, applied side of these approaches I am less familiar with. The researchers at Nuance, who act to create \textit{usable} and \textit{deployable} technologies, would have much to teach me.

To better see what kind of work I have done and what I'm interested in, I think my GitHub account is a good representation:

\begin{center}
\href{<https://github.com/jrmeyer>}{https://github.com/jrmeyer}\\
\end{center}

\vspace{2cm}






\begin{center}
\textit{Thank you for your time and consideration.}  
\end{center}
\end{document}



 
